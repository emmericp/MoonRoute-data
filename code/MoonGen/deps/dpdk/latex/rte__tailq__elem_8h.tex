\hypertarget{rte__tailq__elem_8h}{}\section{lib/librte\+\_\+eal/common/include/rte\+\_\+tailq\+\_\+elem.h File Reference}
\label{rte__tailq__elem_8h}\index{lib/librte\+\_\+eal/common/include/rte\+\_\+tailq\+\_\+elem.\+h@{lib/librte\+\_\+eal/common/include/rte\+\_\+tailq\+\_\+elem.\+h}}


\subsection{Detailed Description}
This file contains the type of the tailq elem recognised by D\+P\+D\+K, which can be used to fill out an array of structures describing the tailq.

In order to populate an array, the user of this file must define this macro\+: rte\+\_\+tailq\+\_\+elem(idx, name). For example\+:


\begin{DoxyCode}
\textcolor{keyword}{enum} rte\_tailq\_t \{
\textcolor{preprocessor}{#define rte\_tailq\_elem(idx, name)     idx,}
\textcolor{preprocessor}{#define rte\_tailq\_end(idx)            idx}
\textcolor{preprocessor}{#include <\hyperlink{rte__tailq__elem_8h}{rte\_tailq\_elem.h}>}
\};

\textcolor{keyword}{const} \textcolor{keywordtype}{char}* rte\_tailq\_names[RTE\_MAX\_TAILQ] = \{
\textcolor{preprocessor}{#define rte\_tailq\_elem(idx, name)     name,}
\textcolor{preprocessor}{#include <\hyperlink{rte__tailq__elem_8h}{rte\_tailq\_elem.h}>}
\};
\end{DoxyCode}


Note that this file can be included multiple times within the same file. 